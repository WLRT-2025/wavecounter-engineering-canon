\documentclass[11pt]{article}

\usepackage[T1]{fontenc}
\usepackage[utf8]{inputenc}

\usepackage{lmodern}
\usepackage{geometry}
\geometry{margin=1in}

\usepackage{amsmath, amssymb}
\usepackage{longtable}

\title{WaveCounter Engineering Canon v1.0\\
\large Core-only Specification (WCCS + WaveCounter A.D.E Core)}
\author{Petr Popov}
\date{2026}

\begin{document}
\maketitle

\section{Object Definition}

WaveCounter Engineering Canon v1.0 defines the structural core of a deterministic and finite market configuration system.

The WaveCounter Canonical Configuration System (WCCS) is formally specified as a deterministic finite automaton (DFA) operating on a closed and finite set of structural configurations.

The system:

\begin{itemize}
\item operates exclusively on canonical configuration sets,
\item is fully deterministic,
\item contains no probabilistic or interpretative branches,
\item is structurally closed under reduction,
\item admits no infinite hierarchical expansion,
\item reaches maximal sufficient hierarchy at level S3.
\end{itemize}

This document discloses only the canonical structural layer.

The applied implementation layer (WaveCounter Structural Engine, WCSE), including parameterization, execution logic, optimization rules, and signal-generation mechanisms, is not part of this specification.
\subsection*{Formal Positioning}

For the purpose of structural priority, WCCS is defined as a closed canonical system with explicitly enumerated configuration sets and a finite transition relation.

All admissible structural states and transitions are exhaustively listed in this document.

Any configuration or transition not explicitly listed is considered invalid within the canonical system.
\section{Canonical Configuration Sets}
% (We will insert E, X, and closure statements here.)
\subsection{Entry-4 Set (|E| = 10)}

The canonical Entry-4 configuration set is defined as:

\[
E =
\{
2143, 2413, 2431,
1324, 3124, 4213,
4231, 1342, 3142, 3412
\}
\]

These are the only admissible four-point structural configurations.

Any four-point configuration not belonging to this set is invalid within WCCS.

\subsection{Exit-5 Set (|X| = 32)}

The canonical Exit-5 configuration set is defined as:

\[
\begin{aligned}
X = \{&
13254, 31254, \\
&13524, 31524, 35124, \\
&13542, 35142, 31542, 35412, \\
&21435, 24135, 24315, 24351, \\
&42135, 42315, 42351, \\
&53124, 51324, 15324, \\
&53412, 53142, 51342, 15342, \\
&24531, 24153, 24513, 21453, \\
&42531, 42513, 42153, \\
&45231, 45213
\}
\end{aligned}
\]

These are the only admissible five-point structural exit configurations.

Any Exit-5 configuration not belonging to this set is invalid within WCCS.


\subsection{Reduction and Closure}

For every admissible transition, reduction produces a valid four-point configuration belonging to the Entry-4 set.

Formally:

\[
R = E
\]

Thus, the system is structurally closed under reduction.

No reduced configuration may fall outside the canonical Entry-4 set.

\section{Canonical Transition Table}
% (We will insert Canonical Table A here.)
The canonical transition relation consists of exactly 32 admissible triples of the form:

\[
(e_i, x_k, r_j)
\]

where:

\begin{itemize}
\item $e_i \in E$ is an Entry-4 configuration,
\item $x_k \in X$ is a corresponding Exit-5 configuration,
\item $r_j \in E$ is the resulting Reduced-4 configuration.
\end{itemize}

The transition function is fully defined by Canonical Table A below.

Any transition not explicitly listed is invalid within WCCS.

\begin{longtable}{lll}
\caption{Canonical Table A (Core-only): Admissible Transitions (Entry-4 $\rightarrow$ Exit-5 $\rightarrow$ Reduced-4)}
\label{tab:canonical_table_a}\\
\hline
\textbf{Entry-4} & \textbf{Exit-5} & \textbf{Reduced-4} \\
\hline
\endfirsthead

\hline
\textbf{Entry-4} & \textbf{Exit-5} & \textbf{Reduced-4} \\
\hline
\endhead

\hline
\endfoot

\hline
\endlastfoot

2143 & 13254 & 1324 \\
2143 & 31254 & 3124 \\

2413 & 13524 & 1324 \\
2413 & 31524 & 3124 \\
2413 & 35124 & 3124 \\

2431 & 13542 & 1342 \\
2431 & 35142 & 3142 \\
2431 & 31542 & 3142 \\
2431 & 35412 & 3412 \\

1324 & 21435 & 2143 \\
1324 & 24135 & 2413 \\
1324 & 24315 & 2431 \\
1324 & 24351 & 2431 \\

3124 & 42135 & 4213 \\
3124 & 42315 & 4231 \\
3124 & 42351 & 4231 \\

4213 & 53124 & 3124 \\
4213 & 51324 & 1324 \\
4213 & 15324 & 1324 \\

4231 & 53412 & 3412 \\
4231 & 53142 & 3142 \\
4231 & 51342 & 1324 \\
4231 & 15342 & 1324 \\

1342 & 24531 & 2431 \\
1342 & 24153 & 2413 \\
1342 & 24513 & 2413 \\
1342 & 21453 & 2143 \\

3142 & 42531 & 4231 \\
3142 & 42513 & 4213 \\
3142 & 42153 & 4213 \\

3412 & 45231 & 4231 \\
3412 & 45213 & 4213 \\

\end{longtable}

\section{Formal DFA Definition}
% (We will insert the formal DFA tuple here.)
The WaveCounter Canonical Configuration System is formally defined as a deterministic finite automaton:

\[
WCCS = (S, \Sigma, \delta, s_0)
\]

where:

\begin{itemize}
\item $S = E$ is the finite set of admissible structural states,
\item $\Sigma = X$ is the finite set of admissible exit events,
\item $\delta : S \times \Sigma \rightarrow S$ is the deterministic transition function,
\item $s_0 \in E$ is an admissible initial structural configuration.
\end{itemize}

The transition function $\delta$ is exhaustively defined by Canonical Table~\ref{tab:canonical_table_a}.

For any $(e, x) \notin T$, the transition $\delta(e,x)$ is undefined and therefore invalid within WCCS.

\section{Structural Invariants}
% (We will insert invariants here.)

The following structural properties hold:

\begin{enumerate}

\item \textbf{Finiteness.}  
The sets $E$ and $X$ are finite.  
The transition relation $T$ is finite and exhaustively listed in Canonical Table~\ref{tab:canonical_table_a}.  
Therefore, the automaton $WCCS$ is finite.

\item \textbf{Determinism.}  
For every admissible pair $(e, x)$ appearing in Table A, the resulting reduced configuration is uniquely determined.

\item \textbf{Closure.}  
All reductions produce configurations belonging to $E$.  
No transition leads outside the canonical state set.

\item \textbf{Non-Extendability.}  
No additional Entry-4, Exit-5, or transition may be introduced without violating the canonical structure defined in this document.

\item \textbf{Hierarchy Bound.}  
The structural hierarchy of WCCS does not extend beyond level S3.  
Additional hierarchical layers do not increase expressive capacity of the canonical configuration system.

\end{enumerate}

\section{Disclosure Boundary}
% (We will insert boundary note here.)
This document discloses the canonical structural layer of WaveCounter only.

The following elements are explicitly excluded from this publication:

\begin{itemize}
\item parameterization schemes,
\item execution logic,
\item optimization procedures,
\item filtering mechanisms,
\item timing rules,
\item applied trading strategies,
\item signal-generation algorithms,
\item performance-related adaptations.
\end{itemize}

These elements belong to the proprietary implementation layer known as
WaveCounter Structural Engine (WCSE) and are not part of the canonical specification.

The present disclosure establishes structural priority only.
\section{Legal Notice}
% (We will insert the All-rights-reserved notice here.)

WaveCounter Engineering Canon v1.0 \\
Copyright \textcopyright\ 2026 Petr Popov. \\
All rights reserved.

This document establishes structural and conceptual priority for the
WaveCounter Canonical Configuration System (WCCS) and WaveCounter A.D.E Core.

No part of this system may be reproduced, redistributed, reverse-engineered,
commercially implemented, or incorporated into derivative commercial systems
without explicit written permission of the author.

The structural specification is publicly disclosed for priority protection purposes.
The applied implementation layer (WaveCounter Structural Engine, WCSE) remains proprietary.

\end{document}